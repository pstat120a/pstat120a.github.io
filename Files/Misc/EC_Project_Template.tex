\documentclass[11pt]{article}
\usepackage[margin = 1in]{geometry}

%~~ TITLE INFORMATION; EDIT TO INCLUDE YOUR INFORMATION
\title{Your Title Here}
\author{List your Teammates Here, Separated by Commas}
\date{\today}
%~~~


%~~ PREAMBLE; NO NEED TO EDIT ANY OF THIS
\usepackage{amsmath}        % basic math package
\usepackage{amssymb}        % basic math package
\usepackage{amsfonts}       % basic math package
\usepackage{dsfont}         % used for additional math support
\usepackage{enumitem}       % used for slightly nicer enumerated lists
\usepackage{hyperref}       % used for hyperlinks

\newcommand{\Prob}{\mathds{P}}      % Probability measure symbol
\newcommand{\E}{\mathds{E}}         % Expectation operator
\newcommand{\Var}{\mathrm{Var}}     % Variance operator
\newcommand{\Cov}{\mathrm{Cov}}     % Covariance operator
\newcommand{\Corr}{\mathrm{Corr}}   % Correlation operator
\newcommand{\1}{\mathds{1}}         % Indicator symbol

\newcommand{\R}{\mathds{R}}     % set of real numbers
\newcommand{\Z}{\mathds{Z}}     % set of integers
\newcommand{\N}{\mathds{N}}     % set of natural numbers

\setlength{\parindent}{0pt}     % undo automatic indentation of first line

\hypersetup{                    % aesthetic setup for hyperlinks
    colorlinks = true,
    linkcolor = blue,
    urlcolor = cyan
}
    
%~~



\begin{document}

\maketitle

%%~~ THIS IS WHERE THE DOCUMENT BEGINS. REPLACE TEXT BELOW THIS LINE WITH WHATEVER YOU NEED.


\section{Glossary of Terms}

To make an enumerated list with Arabic numerals, you can use

\begin{enumerate}[label = (\arabic*)]

    \item First item
    \item Second item
    
\end{enumerate}

You can also make a list with boldfaced Roman numerals, and tighter spacing:

\begin{enumerate}[label = \textbf{(\roman*)}, itemsep = 0mm]

    \item First item
    \item Second item
    
\end{enumerate}



\section{Important Formulae}

Inline equations can be enclosed by single dollar signs: $\cos(x + y)$. (If you need to write an actual dollar sign, use the \textbackslash \  key right before the dollar sign: \$). Display style equations can be enclosed in double dollar signs, like
$$ \Prob(X \in A) = \int_{A} f_X(x) \ \mathrm{d}x  $$
Multi-line equations can be neatly aligned using the \verb|align*| environment:
\begin{align*}
    (x + y)^3   & = \sum_{k=0}^{3} \binom{3}{k} x^k y^{3 - k}   \\
                & = \binom{3}{0} x^0 y^{3 - 0} + \binom{3}{1} x^1 y^{3 - 1} + \binom{3}{2} x^2 y^{3 - 2} + \binom{3}{3} x^3 y^{3 - 3}  \\
                & = y^3 + 3 x y^2 + 3 x^2 y + x^3  
\end{align*}



\section{Three Homework Problems}

\textbf{Boldface} text should be enclosed in a \verb|\boldface{}| environment; \emph{italicized} text can be enclosed in a \verb|\emph{}| environment, and \underline{underlined} text can be enclosed in a \verb|\underline{}| environment.



\section{Three Additional Problems}

If you know you will be referencing a particular equation a lot, I recommend using the \verb|equation| environment:
\begin{equation}{ \label{eq:eq1}
    \E[g(X, Y)]   = \iint_{\R^2} g(x, y) f_{X, Y}(x, y) \ \mathrm{d}A
}\end{equation}
In this way, the equation not only gets a number associated with it but you can automatically refer back to it using the \verb|\ref{}| command, like (\ref{eq:eq1}) [though you may need to typeset your document twice in order for the numbers to show up properly].


\section{Works Cited}

If you would like to hyperlink text, you can use the \verb|\href{}{}| command, like \href{https://pstat120a.github.io}{this}.

\end{document}